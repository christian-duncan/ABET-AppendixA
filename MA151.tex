\label{MA151}  % Replace course code with ID - e.g. \label{CSC110}
\begin{enumerate}[1.]
\item Course number and name\\
  {\bfseries
    % Add course number and name
    % --------------------------
    MA151, Calculus I    
  }
  
\item Credits and contact hours\\
  {\bfseries
    % Add Number of Credits and Contact Hours
    % ---------------------------------------
    4  % Default, change if needed
  }

\item Instructor's or course coordinator's name\\
  {\bfseries
    % Add Instructor name
    % -------------------
    Lisa Hollman    
  }

\item Text book, title, author, and year\\
  {\bfseries
    % Add Textbook info
    % --------------------------
    {\em Single Variable Calculus - Early Transcendentals, 9th edition}, James Stewart, Daniel Clegg, and Saleem Watson, 2020
  }
\begin{enumerate}[a.]
\item Other supplemental materials\\
  {\bfseries
    % List any other supplemental material
    % ------------------------------------
    A graphing calculator (TI-83 or TI-84 recommended)
  }
\end{enumerate}

\item Specific course information
\begin{enumerate}[a.]  
\item Catalog description of the content of the course\\
  {\bfseries
    % Add catalog description
    % -----------------------
This course covers functions and graphs, limits and continuity, derivatives, applications of derivatives, antiderivatives and definite integrals, the Fundamental Theorem of Calculus, numerical integration and applications of definite integrals. A graphing calculator is required; the TI-83 or TI-84 is recommended. Students cannot receive credit for both MA 151 and MA 141.    
  }

\item prerequisites or co-requisites\\
  {\bfseries
    % List any pre/co requisites
    % --------------------------
    % None   % Uncomment if none
    Prerequisites: MA140 (Minimum Grade of C-) or score of 5 on Math Placement Exam  % Just some defaults
    % Corequisites: 
  }

\item indicate whether a required, elective, or selected elective\\ % (as per Table 5-1)
  {\bfseries
    % R, E, or S (uncomment)
    % --------------------------
    Required (as an alternate to MA141)
    % Elective
    % Selected elective
  }

\end{enumerate}

\item Specific goals for the course
\begin{enumerate}
\item specific outcomes of instruction\\ % , for example, ``The student will be able to explain the significance of current research about a particular topic.'' \\
  {\bfseries
    % 
    % --------------------------
    Course Learning Objectives (the student will be able to):
    \begin{enumerate}[1.]
\item Evaluate limits, derivatives, and basic integrals.
\item Use derivatives to solve several varieties of problems.
\item Understand the meaning of the derivative in terms of rate of change.
\item Understand the meaning of the derivative as a limit.
\item Understand the meaning of the definite integral in terms of a limit.
\item Understand the meaning of the definite integral in terms of area.
\item Understand the relationship between the derivative and the definite integral as expressed in the Funda-
mental Theorem of Calculus.
\item Calculate elementary integrals.
\item Relate calculus concepts to the graphical, numerical, and symbolic representations of functions.
\item Solve a wide variety of problems from physics, engineering, and mathematics.
\item Appreciate calculus as a tool for modeling reality.
    \end{enumerate}
  }

\item explicitly indicate which of the student outcomes listed in Criterion 3 or any other outcomes are addressed by the course.\\
  {\bfseries
    None
    % C1 (CLO ...),
    % C2 (CLO ...),
    % C3 (CLO ...),
    % C4 (CLO ...),
    % C5 (CLO ...),
    % C6 (CLO ...)
  }
\end{enumerate}

\item Brief list of topics to be covered\\
  {\bfseries
    Lecture Topics are not specified but see objectives which gives a detailed topic list.
  }

\end{enumerate}

\noindent Prepared by: Christian Duncan\\
\noindent Creation date: 06/26/2021\\
\noindent Revised:\\
