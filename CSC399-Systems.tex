\label{CSC399-Systems}  % Replace course code with ID - e.g. \label{CSC110}
\begin{enumerate}[1.]
\item Course number and name\\
  {\bfseries
    % Add course number and name
    % --------------------------
    CSC399, Systems Programming
  }
  
\item Credits and contact hours\\
  {\bfseries
    % Add Number of Credits and Contact Hours
    % ---------------------------------------
    3    
  }

\item Instructor's or course coordinator's name\\
  {\bfseries
    % Add Instructor name
    % -------------------
    Christian Duncan    
  }

\item Text book, title, author, and year\\
  {\bfseries
    % Add Textbook info
    % --------------------------
    {\em A Practical Guide to Ubuntu Linux (3rd Edition)}, by Mark Sobel.
    }
\begin{enumerate}[a.]
\item Other supplemental materials\\
  {\bfseries
    % List any other supplemental material
    % ------------------------------------
    None. (Instructor will work with student to identify a good C programming book or online source based on their experience.)
  }
\end{enumerate}

\item Specific course information
\begin{enumerate}[a.]  
\item Catalog description of the content of the course\\
  {\bfseries
    % Add catalog description
    % -----------------------
The purpose of this course is to provide the students with an introduction to system-level programming.
Although not the primary focus of this course, instruction shall be done
within the context of C and Linux/FreeBSD.
}

\item prerequisites or co-requisites\\
  {\bfseries
    % List any pre/co requisites
    % --------------------------
    % None   % Uncomment if none
    Prerequisites: CSC215, SER225 (Minimum Grade C-)
    % Corequisites: 
  }

\item indicate whether a required, elective, or selected elective\\ % (as per Table 5-1)
  {\bfseries
    % R, E, or S (uncomment)
    % --------------------------
    % Required
    % Elective
    Selected elective
  }

\end{enumerate}

\item Specific goals for the course
\begin{enumerate}
\item specific outcomes of instruction\\ % , for example, ``The student will be able to explain the significance of current research about a particular topic.'' \\
  {\bfseries
    % 
    % --------------------------
    Course Learning Outcomes (the student will be able to):
\begin{enumerate}[1.]
\item To work effectively in a UNIX-style environment.
\item To explain the basic operations that are performed from the time a computer is turned on until a user is able to execute programs.
\item To write medium to large C programs for a range of applications.
\item To use systems tools for C programming.
\item To write C programs that use the UNIX system call interface.
\item To write small to medium size scripts, in various scripting languages, for a range of applications. 
\end{enumerate}
  }

\item explicitly indicate which of the student outcomes listed in Criterion 3 or any other outcomes are addressed by the course.\\
  {\bfseries
    % None
    % C1 (CLO 3, 5, 6)
    C2 (CLO 3, 4, 5, 6),
    % C3 (CLO 5),
    % C4 (CLO 4),
    % C5 (CLO 5)
    C6 (CLO 3, 4, 5, 6)
  }
\end{enumerate}

\item Brief list of topics to be covered\\
  {\bfseries
    Lecture Topics
    \begin{itemize}
    \item Using standard Linux desktop user environments, file systems, and tools.
    \item Using the command line to interact with the Linux Operating system.
    \item Using advanced shell commands such as piping, I/O redirects, and variable substitution.
    \item Writing programs in a scripting language (Bash).
    \item Writing programs in the C programming language, including using pointers and memory management.
    \item Using standard C libraries for various programming tasks.
    \item Using various tools to enhance programming, such as makefiles, profilers, lint, and debuggers.
    \item Examining what happens during program compilation, linking, and loading.
    \item Interacting directly with the operating system by making system calls for file management, file execution, process control, and interprocess communication.
    \item Main project: Implement a simple interactive shell (similar to Bash or TCSH)
    \end{itemize}
  }
\end{enumerate}

\noindent Prepared by: Christian Duncan\\
\noindent Creation date: 06/28/2021\\
\noindent Revised:\\
