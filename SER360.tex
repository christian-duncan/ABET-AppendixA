\label{SER360}  % Replace course code with ID - e.g. \label{CSC110}
\begin{enumerate}[1.]
\item Course number and name\\
  {\bfseries
    % Add course number and name
    % --------------------------
    SER 360, Software Engineering in Healthcare
  }
  
\item Credits and contact hours\\
  {\bfseries
    % Add Number of Credits and Contact Hours
    % ---------------------------------------
    3  % Default, change if needed
  }

\item Instructor's or course coordinator's name\\
  {\bfseries
    % Add Instructor name
    % -------------------
    Stefan Christov
  }

\item Text book, title, author, and year\\
  {\bfseries
    % Add Textbook info
    % --------------------------
    Biomedical Informatics: Computer Applications in Health Care and Biomedicine, Edward Shortliffe and James Cimino, 2014
  }
\begin{enumerate}[a.]
\item Other supplemental materials\\
  {\bfseries
    % List any other supplemental material
    % ------------------------------------
    
  }
\end{enumerate}

\item Specific course information
\begin{enumerate}[a.]  
\item Catalog description of the content of the course\\
  {\bfseries
    % Add catalog description
    % -----------------------
    Biomedical informatics is one of the fastest growing economic sectors in the world. Software, and thus software engineering, has an important role in biomedical informatics. Students in this course explore the applicability of software engineering techniques to healthcare. Topics include electronic health records; modeling and analysis of medical processes with the goal of improving safety and efficiency; software solutions for providing clinical decision support; and bioinformatics.
  }

\item prerequisites or co-requisites\\
  {\bfseries
    % List any pre/co requisites
    % --------------------------
    % None   % Uncomment if none
    Prerequisites: CSC215, SER225 (Minimum Grade C-) \\  % Just some defaults
    % Corequisites: 
  }

\item indicate whether a required, elective, or selected elective\\ % (as per Table 5-1)
  {\bfseries
    % R, E, or S (uncomment)
    % --------------------------
    % Required
    Elective
    % Selected elective
  }

\end{enumerate}

\item Specific goals for the course
\begin{enumerate}
\item specific outcomes of instruction\\ % , for example, ``The student will be able to explain the significance of current research about a particular topic.'' \\
  {\bfseries
    % 
    % --------------------------
    Course Learning Objectives (the student will be able to):
    \begin{itemize}
      \item Describe the major challenges and issues in the field of biomedical informatics and the role of software engineering in addressing them
      \item Analyze electronic health records with respect to core functionalities they need to provide
      \item Model medical processes using a notation with well-defined semantics
      \item Apply software engineering approaches to analyze medical processes to improve their safety and efficiency
      \item Design and implement a simple clinical decision support system
      \item Describe bioinformatics and apply basic computational techniques utilized in that field
      \item Describe potential future directions for integrating software engineering in healthcare
    \end{itemize}
  }

\item explicitly indicate which of the student outcomes listed in Criterion 3 or any other outcomes are addressed by the course.\\
  {\bfseries
    Other outcomes addressed:\\
    An ability to identify, formulate, and solve complex engineering problems by applying principles of engineering, science, and mathematics (CLOs 1-6)\\
    An ability to apply engineering design to produce solutions that meet specified needs with consideration of public health, safety, and welfare, as well as global, cultural, social, environmental, and economic factors (CLO 5)\\
   An ability to communicate effectively with a range of audiences (CLOs 1, 5, 7)\\
   An ability to function effectively on a team whose members together provide leadership, create a collaborative and inclusive environment, establish goals, plan tasks, and meet objectives (CLO 5)

    % C1 (CLO ...),
    % C2 (CLO ...),
    % C3 (CLO ...),
    % C4 (CLO ...),
    % C5 (CLO ...),
    % C6 (CLO ...)
  }
\end{enumerate}

\item Brief list of topics to be covered\\
  {\bfseries
    Lecture Topics
    \begin{itemize}
      \item Role of computing in health care
      \item Electronic health records
      \item Medical errors and processes as software
      \item Process modeling
      \item Property specification
      \item Applying model checking to verify properties of medical processes 
      \item Fault tree analysis
      \item Clinical decision support
      \item Bioinformatics and sequence analysis
      \item Student presentations of a topic related to the role of computing in health care
      \item Course project: in collaboration with nursing students and professors, specify, design, implement, and evaluate a software system for documenting a cardiac arrest
    \end{itemize}
  }

\end{enumerate}

\noindent Prepared by: Stefan Christov\\
\noindent Creation date: 06/29/2021\\
\noindent Revised:\\
