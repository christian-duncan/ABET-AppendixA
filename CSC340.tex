\label{CSC340}  % Replace course code with ID - e.g. \label{CSC110}
\begin{enumerate}[1.]
\item Course number and name\\
  {\bfseries
    % Add course number and name
    % --------------------------
    CSC340, Networking and Distributed Processing
  }
  
\item Credits and contact hours\\
  {\bfseries
    % Add Number of Credits and Contact Hours
    % ---------------------------------------
    3    
  }

\item Instructor's or course coordinator's name\\
  {\bfseries
    % Add Instructor name
    % -------------------
    Christian Duncan    
  }

\item Text book, title, author, and year\\
  {\bfseries
    % Add Textbook info
    % --------------------------
{\em Computer Networks, 5th Edition}, Tanenbaum and Wetherall, 2010
  }
\begin{enumerate}[a.]
\item Other supplemental materials\\
  {\bfseries
    % List any other supplemental material
    % ------------------------------------
    None    
  }
\end{enumerate}

\item Specific course information
\begin{enumerate}[a.]  
\item Catalog description of the content of the course\\
  {\bfseries
    % Add catalog description
    % -----------------------
This course introduces students to net-centric computing, the web as an example of client-server computing, building internet and web applications, communications and networking, distributed object systems, collaboration technology and groupware, distributed operating systems and distributed systems.
}

\item prerequisites or co-requisites\\
  {\bfseries
    % List any pre/co requisites
    % --------------------------
    % None   % Uncomment if none
    Prerequisites: CSC215, SER225 (Minimum Grade C-)
    % Corequisites: 
  }

\item indicate whether a required, elective, or selected elective\\ % (as per Table 5-1)
  {\bfseries
    % R, E, or S (uncomment)
    % --------------------------
    Required
    % Elective
    % Selected elective
  }

\end{enumerate}

\item Specific goals for the course
\begin{enumerate}
\item specific outcomes of instruction\\ % , for example, ``The student will be able to explain the significance of current research about a particular topic.'' \\
  {\bfseries
    % 
    % --------------------------
    Course Learning Outcomes (the student will be able to):
\begin{enumerate}[1.]
\item Explain inherent challenges in network communications;
\item Describe the primary layers of computer networks;
\item Apply the standard protocols that form the basis of the Internet;
\item Explain legal and ethical principles around network communications;
\item Develop, as a team, basic multi-threaded network applications; and
\item Use distributed computing to solve computational problems.
\end{enumerate}
  }

\item explicitly indicate which of the student outcomes listed in Criterion 3 or any other outcomes are addressed by the course.\\
  {\bfseries
    % None
    C1 (CLO 3, 5, 6)
    C2 (CLO 5, 6),
    % C3 (CLO 5),
    C4 (CLO 4),
    C5 (CLO 5)
    C6 (CLO 5, 6)
  }
\end{enumerate}

\item Brief list of topics to be covered\\
  {\bfseries
    Lecture Topics
    \begin{itemize}
    \item The layers of a Network
    \item Network applications including DNS, Email, and the Web
    \item Threading (for Distributed Processing in Networks)
    \item Client-Server Model
    \item Transport Layer (TCP): sockets, sequence numbering, connection request (three-way handshake), connection release (two generals' problem), flow control, and packet parsing
    \item Network Layer (IP): routing algorithms, congestion control, internet protocol (IPv4/IPv6), internetworking,
    \item Serialization
    \item Designing and developing a multi-player networked game 
    % \item Physical Layer % If in-person, sigh.
    \end{itemize}
  }
\end{enumerate}

\noindent Prepared by: Christian Duncan\\
\noindent Creation date: 06/28/2021\\
\noindent Revised:\\
