\label{CSC315}  % Replace course code with ID - e.g. \label{CSC110}
\begin{enumerate}[1.]
\item Course number and name\\
  {\bfseries
    % Add course number and name
    % --------------------------
    CSC315, Theory of Computation
  }
  
\item Credits and contact hours\\
  {\bfseries
    % Add Number of Credits and Contact Hours
    % ---------------------------------------
    3    
  }

\item Instructor's or course coordinator's name\\
  {\bfseries
    % Add Instructor name
    % -------------------
    Christian Duncan    
  }

\item Text book, title, author, and year\\
  {\bfseries
    % Add Textbook info
    % --------------------------
    {\em Theory of Computation, 3rd Edition}, Sipser, 2012
  }
\begin{enumerate}[a.]
\item Other supplemental materials\\
  {\bfseries
    % List any other supplemental material
    % ------------------------------------
    None    
  }
\end{enumerate}

\item Specific course information
\begin{enumerate}[a.]  
\item Catalog description of the content of the course\\
  {\bfseries
    % Add catalog description
    % -----------------------
This course provides an introduction to the classical theory of computer science.
In particular, it covers automata, formal languages, computability, and complexity.

Our aim in this course is to develop a fundamental understanding of the nature of computing.
Throughout all the topics, we will answer one underlying question stated on page 1 of the course textbook:
\begin{quote}
``What are the fundamental capabilities and limitations of computers?''
\end{quote}
We will look at sub-questions like: ``What can be computed?''; ``How do you prove something
cannot be computed?  (Is this even possible?)''; ``What makes some problems so much harder
than others to solve?''
  }

\item prerequisites or co-requisites\\
  {\bfseries
    % List any pre/co requisites
    % --------------------------
    % None   % Uncomment if none
    Prerequisites: CSC215 or MA301 (Minimum Grade C-)
    % Corequisites: 
  }

\item indicate whether a required, elective, or selected elective\\ % (as per Table 5-1)
  {\bfseries
    % R, E, or S (uncomment)
    % --------------------------
    Required
    % Elective
    % Selected elective
  }

\end{enumerate}

\item Specific goals for the course
\begin{enumerate}
\item specific outcomes of instruction\\ % , for example, ``The student will be able to explain the significance of current research about a particular topic.'' \\
  {\bfseries
    % 
    % --------------------------
    Course Learning Outcomes (the student will be able to):
\begin{enumerate}[1.]
\item Create finite state and push-down automata with specific properties (to recognize specific languages).
\item Create regular and context-free grammars with specific properties (to generate specific languages).
\item Use the pumping lemma to show particular problems cannot be solved by finite state automata (particular languages are not regular).
\item Use the pumping lemma to show particular problems cannot be solved by push-down automata (particular languages are not context-free).
\item Create Turing Machines to solve particular problems (to recognize specific languages).
\item Use diagonalization or reducibility methods to prove a problem is undecidable.
\item Explain the importance of NP-completeness and the ``P=NP'' problem.
\item Use reducibility to prove a problem is NP-complete.
\item Explain the relationship between deterministic and non-deterministic computation time and space.
\end{enumerate}
  }

\item explicitly indicate which of the student outcomes listed in Criterion 3 or any other outcomes are addressed by the course.\\
  {\bfseries
    % None
    C1 (CLO 1, 2, 5),
    % C2 (CLO ),
    C3 (CLO 3, 4, 6, 7, 8, 9),
    % C4,
    % C5,
    C6 (CLO 1, 2, 5)
  }
\end{enumerate}

\item Brief list of topics to be covered\\
  {\bfseries
    Lecture Topics
    \begin{itemize}
    \item Mathematics Review
    \item Regular Languages
    \item Context-Free Languages
    \item Church-Turing Thesis (Turing Machines)
    \item Decidability
    \item Reducibility
    \item Time Complexity
    \item NP-Completeness
    \end{itemize}
  }

\end{enumerate}

\noindent Prepared by: Christian Duncan\\
\noindent Creation date: 06/28/2021\\
\noindent Revised:\\
