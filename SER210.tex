\label{SER210}  % Replace course code with ID - e.g. \label{CSC110}
\begin{enumerate}[1.]
\item Course number and name\\
  {\bfseries
    % Add course number and name
    % --------------------------
SER 210, Software Engineering Design and Development
  }

\item Credits and contact hours\\
  {\bfseries
    % Add Number of Credits and Contact Hours
    % ---------------------------------------
3
  }

\item Instructor's or course coordinator's name\\
  {\bfseries
    % Add Instructor name
    % -------------------
Prof. Ruby El Kharboutly
  }

\item Text book, title, author, and year\\
  {\bfseries
    % Add Textbook info
    % --------------------------
    Head First Android Development by Dawn Griffiths, David Griffiths (2015)
  }
\begin{enumerate}[a.]
\item Other supplemental materials\\
  {\bfseries
    % List any other supplemental material
    % ------------------------------------
    None
  }
\end{enumerate}

\item Specific course information
\begin{enumerate}[a.]
\item Catalog description of the content of the course\\
  {\bfseries
    % Add catalog description
    % -----------------------
This course serves as an introduction to software engineering using object-oriented analysis and
design. The course emphasizes the development of robust and high-quality software systems based on
object-oriented principles. Implementations are performed using state-of-the-art programming
languages and application development frameworks.
  }

\item prerequisites or co-requisites\\
  {\bfseries
    % List any pre/co requisites
    % --------------------------
    % None   % Uncomment if none
    Prerequisite: SER120; SER120L (minimum grade of C-)
  }

\item indicate whether a required, elective, or selected elective\\ % (as per Table 5-1)
  {\bfseries
    % R, E, or S (uncomment)
    % --------------------------
    % Required
    % Elective
    Selected Elective
  }

\end{enumerate}

\item Specific goals for the course
\begin{enumerate}
\item specific outcomes of instruction\\ % , for example, ``The student will be able to explain the significance of current research about a particular topic.'' \\
  {\bfseries
    %
    % --------------------------
    Learning Objectives (the student will be able to):
    \begin{itemize}
      \item Understand how Android applications work, their life cycle, manifest, Intents and using external
resources
      \item Design and develop Android apps with compelling user interfaces using layouts, views,
fragments, action bar and menus
      \item Create apps that support networking, threading and background processing
      \item Store and manipulate data using Shared Preferences and SQLite.
      \item Use Android Studio, Gradle and the Android SDK for development
      \item Follow the software engineering process form specification to testing
      \item Demonstrate an understanding of software engineering fundamentals
    \end{itemize}
  }

\item explicitly indicate which of the student outcomes listed in Criterion 3 or any other outcomes are addressed by the course.\\
  {\bfseries
    None
    % C1,
    % C2,
    % C3,
    % C4,
    % C5,
    % C6
  }
\end{enumerate}

\item Brief list of topics to be covered\\
  {\bfseries
    Lecture Topics
    \begin{itemize}
      \item Android Basics
      \item Activities and Intents
      \item Activity life cycle
      \item UI and Layouts
      \item List view and Adapters
      \item Networking Threading and Permissions
      \item Fragments
      \item Navigation
      \item Intro to Software Engineering
      \item Requirements Analysis
      \item Cursors
      \item Design and Project Management
      \item Testing
      \item Content providers and Security
    \end{itemize}
  }

\end{enumerate}

\noindent Prepared by: Christian Duncan\\
\noindent Creation date: 06/26/2021\\
\noindent Revised: Jonathan Blake (6/27/2021)\\
