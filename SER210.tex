\label{SER210}  % Replace course code with ID - e.g. \label{CSC110}
\begin{enumerate}[1.]
\item Course number and name\\
  {\bfseries
    % Add course number and name
    % --------------------------
    SER210, Software Engineering Design and Development
  }

\item Credits and contact hours\\
  {\bfseries
    % Add Number of Credits and Contact Hours
    % ---------------------------------------
    3
  }

\item Instructor's or course coordinator's name\\
  {\bfseries
    % Add Instructor name
    % -------------------
    Ruby ElKharboutly
  }

\item Text book, title, author, and year\\
  {\bfseries
    % Add Textbook info
    % --------------------------
	Head First Android Development, 2nd Edition
	Author: Dawn Griffiths, David Griffiths
	ASIN: B0753SBSGT
	O’Reilly , 2017

  }
\begin{enumerate}[a.]
\item Other supplemental materials\\
  {\bfseries
    % List any other supplemental material
    % ------------------------------------
    None
  }
\end{enumerate}

\item Specific course information
\begin{enumerate}[a.]
\item Catalog description of the content of the course\\
  {\bfseries
    % Add catalog description
    % -----------------------
    This course serves as an introduction to the principles of design and programming of software using object-oriented techniques such as inheritance, polymorphism, encapsulation and information hiding. Students will learn event-driven programming and the use of graphics. Additional topics include collections, design patterns and frameworks   }

\item prerequisites or co-requisites\\
  {\bfseries
    % List any pre/co requisites
    % --------------------------
    % None   % Uncomment if none
    Prerequisites: SER 120 (Minimum grade of C-)
    % Corequisites:
  }

\item indicate whether a required, elective, or selected elective\\ % (as per Table 5-1)
  {\bfseries
    % R, E, or S (uncomment)
    % --------------------------
    Elective
    % Elective
    % Selected elective
  }

\end{enumerate}

\item Specific goals for the course
\begin{enumerate}
\item specific outcomes of instruction\\ % , for example, ``The student will be able to explain the significance of current research about a particular topic.'' \\
  {\bfseries
    %
    % --------------------------
    Course Learning Objectives (the student will be able to):
\begin{enumerate}
\item Understand how Android applications work, their life cycle, manifest, Intents and using external resources;
\item Design and develop Android apps with compelling user interfaces using layouts, views, fragments, action bar and menus;
\item Create apps that support networking, threading and background processing;
\item Store and manipulate data using Shared Preferences and SQLite;
\item Use Android Studio, Gradle and the Android SDK for development;
\item Follow the software engineering process form specification to testing;
\item Demonstrate an understanding of software engineering fundamentals
\end{enumerate}
  }

\item explicitly indicate which of the student outcomes listed in Criterion 3 or any other outcomes are addressed by the course.\\
  {\bfseries

    C2 (CLO 1, 2, 3,4),
    C3 (CLO 2,6),
	C6 (CLO 2,4,6,7)
  }
\end{enumerate}

\item Brief list of topics to be covered\\
  {\bfseries
    Lecture Topics
    \begin{itemize}
      \item User Inteface and Layouts
      \item List views and adapters
	  \item Networking and concurrent operation
	  \item Fragment
	  \item Action bar implemenation
	  \item Navigation
	  \item Storage and SQL lite
	  \item Introduction to requirements analysis
	  \item Introduction to software design and project management
	  \item Testing
    \end{itemize}
  }

\end{enumerate}

\noindent Prepared by: Ruby ElKharboutly\\
\noindent Creation date: 06/28/2021\\
\noindent Revised:\\
