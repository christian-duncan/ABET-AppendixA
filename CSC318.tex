\label{CSC318}  % Replace course code with ID - e.g. \label{CSC110}
\begin{enumerate}[1.]
\item Course number and name\\
  {\bfseries
    % Add course number and name
    CSC318, Cryptography (MA318)
  }
  
\item Credits and contact hours\\
  {\bfseries
    % Add Number of Credits and Contact Hours
    % ---------------------------------------
    3  % Default, change if needed
  }

\item Instructor's or course coordinator's name\\
  {\bfseries
    % Add Instructor name
    % -------------------
    David Herscovici    
  }

\item Text book, title, author, and year\\
  {\bfseries
    % Add Textbook info
    % --------------------------
{\em Introduction to Cryptography with Coding Theory, second edition}, Wade Trappe and Lawrence C. Washington, Pearson/Prentice Hall, 2007    
  }
\begin{enumerate}[a.]
\item Other supplemental materials\\
  {\bfseries
    % List any other supplemental material
    % ------------------------------------
    None    
  }
\end{enumerate}

\item Specific course information
\begin{enumerate}[a.]  
\item Catalog description of the content of the course\\
  {\bfseries
    % Add catalog description
    % -----------------------
Students study methods of transmitting information securely in the face of a malicious adversary deliberately trying to read or alter it. Participants also discuss various possible attacks on these communications. Students learn about classical private-key systems, the Data Encryption Standard (DES), the RSA public-key algorithm, discrete logarithms, hash functions and digital signatures. Additional topics may include the Advanced Encryption Standard (AES), digital cash, games, zero-knowledge techniques and information theory, as well as topics chosen by the students together with the instructor for presentations.    
  }

\item prerequisites or co-requisites\\
  {\bfseries
    % List any pre/co requisites
    % --------------------------
    % None   % Uncomment if none
    Prerequisites: MA229 or CSC215 (Minimum Grade C-) 
    % Corequisites: 
  }

\item indicate whether a required, elective, or selected elective\\ % (as per Table 5-1)
  {\bfseries
    % R, E, or S (uncomment)
    % --------------------------
    % Required
    % Elective
    Selected elective\footnote{Students can take this as either a CSC elective or a Math elective, but not both.}
  }

\end{enumerate}

\item Specific goals for the course
\begin{enumerate}
\item specific outcomes of instruction\\ % , for example, ``The student will be able to explain the significance of current research about a particular topic.'' \\
  {\bfseries
    % 
    % --------------------------
    Not specified.
  }

\item explicitly indicate which of the student outcomes listed in Criterion 3 or any other outcomes are addressed by the course.\\
  {\bfseries
    None
    % C1 (CLO ...),
    % C2 (CLO ...),
    % C3 (CLO ...),
    % C4 (CLO ...),
    % C5 (CLO ...),
    % C6 (CLO ...)
  }
\end{enumerate}

\item Brief list of topics to be covered\\
  {\bfseries
    Lecture Topics
    \begin{itemize}
\item Basic framework; Congruences; Modular arithmetic; Shift ciphers; Affine ciphers
\item Greatest Common Divisors (gcd’s); The Euclidean Algorithm and the extended Euclidean algorithm; Modular inverses; Attacking affine ciphers; using programs
\item Vigenere ciphers; Substitution ciphers
\item Matrix operations; the Hill cipher; Binary and hexadeimal representation of numbers; One-time pads
\item Data Encryption Standard (DES): simplified and normal version
\item Attacking DES; Finite fields; the Advanced Encryption Standard (AES)
\item Introduction to the RSA algorithm; Modular Exponentiation; the Chinese Remainder Theorem
\item Fermat’s and Euler’s Theorems; the Euler function;
\item Public key cryptography; Digital signatures
\item Primality testing; factoring methods; the RSA challenge; the Quadratic Sieve
\item Primitive roots; Discrete logarithms; the ElGamal cryptosystem; the Pohlig-Hellman algorithm
\item Hash functions; Birthday attacks
\item The Secure Hash Algorithm (SHA); Diffie-Hellman key exchange;
\end{itemize}
}

\end{enumerate}

\noindent Prepared by: Christian Duncan\\
\noindent Creation date: 06/29/2021\\
\noindent Revised:\\
