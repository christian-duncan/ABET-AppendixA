\label{BIO150}  % Replace course code with ID - e.g. \label{CSC110}
\begin{enumerate}[1.]
\item Course number and name\\
  {\bfseries
    % Add course number and name
    % --------------------------
    BIO150, General Biology for Majors
  }
  
\item Credits and contact hours\\
  {\bfseries
    % Add Number of Credits and Contact Hours
    % ---------------------------------------
    4    
  }

\item Instructor's or course coordinator's name\\
  {\bfseries
    % Add Instructor name
    % -------------------
    Coordinator: Prof. Nancy Burns
  }

\item Text book, title, author, and year\\
  {\bfseries
    % Add Textbook info
    % --------------------------
    \begin{itemize}
    \item {\em A Student Handbook for Writing in Biology, 5th Edition}, Knisely, Karin, 2017.
    \item {\em Mastering Biology Online Study System, 12th Edition} (ISBN 9780135855836),
          Urry, L. A.,
          Cain, M. L.,
          Wasserman, S. A.,
          Minorsky, P. V.,
          Orr, R. B.,
          Campbell, N.A, 2020.
    \end{itemize}
  }
\begin{enumerate}[a.]
\item Other supplemental materials\\
  {\bfseries
    % List any other supplemental material
    % ------------------------------------
    None.
  }
\end{enumerate}

\item Specific course information
\begin{enumerate}[a.]  
\item Catalog description of the content of the course\\
  {\bfseries
    % Add catalog description
    % -----------------------
Students develop sound learning strategies and introductory knowledge within five core concepts in biology: science as a way of knowing, chemistry of life, structure and function relationships; major pathways and transformations of energy and matter, as well as living systems as interactive and interconnected. This is the first course of a three-course sequence for biology and related majors.
  }

\item prerequisites or co-requisites\\
  {\bfseries
    % List any pre/co requisites
    % --------------------------
    % None   % Uncomment if none
    Corequisites: BIO150L
  }

\item indicate whether a required, elective, or selected elective\\ % (as per Table 5-1)
  {\bfseries
    % R, E, or S (uncomment)
    % --------------------------
    % Required
    % Elective
    Selected elective
  }

\end{enumerate}

\item Specific goals for the course
\begin{enumerate}
\item specific outcomes of instruction\\ % , for example, ``The student will be able to explain the significance of current research about a particular topic.'' \\
  {\bfseries
    % 
    % --------------------------
    % Learning Objectives (the student will be able to):\\
    Outcomes are given as a list of topics (see below).
  }

\item explicitly indicate which of the student outcomes listed in Criterion 3 or any other outcomes are addressed by the course.\\
  {\bfseries
    None
    % C1,
    % C2,
    % C3,
    % C4,
    % C5,
    % C6
  }
\end{enumerate}

\item Brief list of topics to be covered\\
  {\bfseries
    Lecture Topics
    \begin{itemize}
      \item Science as a Way of Knowing: To help students to understand major epistemological considerations, e.g., How is science different from other kinds of inquiry, e.g., like faith or other philosophical disciplines? What is the Criterion of Demarcation? What is a hypothesis? What distinguishes treatments and controls? What does the asymmetry of proof and disproof refer to and why is this issue important to understanding what scientific theories are. What are the three hallmarks of a scientific investigation?
      \item Atoms, Bonds and Molecules: Why do atoms interact and form bonds? What kinds of bonds are common in biological systems and what characteristics do they have?
      \item Macromolecules: their Chemistry and Biology: What are the four major kinds of organic molecules, their structural features, and functional roles in biological systems? What kinds of bonds are critical to the functioning of each kind of macromolecule? What are the structural features of nucleic acids and proteins that enable reproduction, information storage, mutation, and catalysis?
      \item Energy, Enzymes and Catalysis: What is catalysis and how is it regulated in biological systems? What are the structural features of biological catalysts that enable them to work with lock-and-key specificity? What are the typical energetics of a catalyzed reaction? What ultimately determines the timing and structure of the various catalysts?
      \item Prokaryotes and Eukaryotes: What are the structural and functional differences between prokaryotes and eukaryotes?
      \item Cell Communication: How do membranes work? How is transport across membranes regulated? What are the components of the endomembrane system and how do they interact? How did the double membranes of the nucleus, mitochondria, and chloroplasts originate - what are the contending hypotheses and evidence?
      \item Respiration and Photosynthesis: How is energy captured and converted to various chemical forms in photosynthesis and respiration? How do photosynthesis and respiration work as biochemical systems, including major inputs and outputs? What is chemiosmosis and how does it function, in both photosynthesis and respiration, and how are membranes and their proteins involved in this work?
      \item Anatomy and Physiology: How are vertebrate systems organized?  What are the major organs in a vertebrate system?  How do these major organs function for homeostasis?  What are the cellular components that control the function of cells, tissues and organs in an organ system?
    \end{itemize}
  }

\end{enumerate}

\noindent Prepared by: Christian Duncan\\
\noindent Creation date: 06/26/2021\\
\noindent Revised:\\
