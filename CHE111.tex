\label{CHE111}  % Replace course code with ID - e.g. \label{CSC110}
\begin{enumerate}[1.]
\item Course number and name\\
  {\bfseries
    % Add course number and name
    % --------------------------
CHE 111, General Chemistry II
  }

\item Credits and contact hours\\
  {\bfseries
    % Add Number of Credits and Contact Hours
    % ---------------------------------------
3
  }

\item Instructor's or course coordinator's name\\
  {\bfseries
    % Add Instructor name
    % -------------------
    Frank Tagliaferro
  }

\item Text book, title, author, and year\\
  {\bfseries
    % Add Textbook info
    % --------------------------
  General Chemistry by D. Ebbing \& S. Gammon, 11th ed., Cengage Learning, 2017
  }
\begin{enumerate}[a.]
\item Other supplemental materials\\
  {\bfseries
    % List any other supplemental material
    % ------------------------------------
Cengage Unlimited, OWLv2, Calculator supporting logarithmic functions
  }
\end{enumerate}

\item Specific course information
\begin{enumerate}[a.]
\item Catalog description of the content of the course\\
  {\bfseries
    % Add catalog description
    % -----------------------
Students study intermolecular forces, properties of solutions, kinetics, chemical equilibrium, pH, acid-base solution chemistry, thermodynamics and electrochemistry. Problem-solving is emphasized.
  }

\item prerequisites or co-requisites\\
  {\bfseries
    % List any pre/co requisites
    % --------------------------
    % None   % Uncomment if none
    Prerequisites: CHE110, CHE110L (Minimum Grade C-)\\
    Corequisites: CHE111L
  }

\item indicate whether a required, elective, or selected elective\\ % (as per Table 5-1)
  {\bfseries
    % R, E, or S (uncomment)
    % --------------------------
    % Required
    % Elective
    Selected elective
  }

\end{enumerate}

\item Specific goals for the course
\begin{enumerate}
\item specific outcomes of instruction\\ % , for example, ``The student will be able to explain the significance of current research about a particular topic.'' \\
  {\bfseries
    %
    % --------------------------
    Outcomes are given as a list of fundamental principles and concepts (see below).
  }

\item explicitly indicate which of the student outcomes listed in Criterion 3 or any other outcomes are addressed by the course.\\
  {\bfseries
    None
    % C1,
    % C2,
    % C3,
    % C4,
    % C5,
    % C6
  }
\end{enumerate}

\item Brief list of topics to be covered\\
  {\bfseries
    Fundamental Principles and Concepts
    \begin{itemize}
      \item Understand the role of intermolecular forces in the physical properties of liquids and solids
      \item Demonstrate a basic knowledge of crystal structure and the impact of intermolecular forces on physical properties
      \item Understand phase changes and develop the ability to interpret phase diagrams
      \item Calculate the concentration of solutions using molarity, molality, and % by mass
      \item Understand the effect of temperature and pressure on solubility
      \item Determine the melting point, freezing point, vapor pressure and osmotic pressure of solutions using colligative properties for solutions composed of non-electrolytes and electrolytes
      \item Determine the rate law and reaction rate of a chemical system
      \item Understand activation energy and how temperature affects the rate of a reaction
      \item Understand reaction mechanisms and the role of catalysts in the rate of a reaction
      \item Understand and apply the principles of equilibrium to chemical systems
      \item Calculate concentrations of products or reactants using the equilibrium expression and equilibrium constants
      \item Apply Le Chatelier’s Principle to chemical systems
      \item Determine the pH of acids, bases, buffers and salt solutions
      \item Understand and apply the Laws of Thermodynamics
      \item Predict whether a chemical reaction will occur spontaneously
      \item Balance oxidation-reduction reactions
      \item Determine the electrical potential of an oxidation-reduction reaction
      \item Understand and use the Nernst Equation to calculate free energy, equilibrium constants and electrical potential of systems that are not at standard conditions
    \end{itemize}
  }

\end{enumerate}

\noindent Prepared by: Christian Duncan\\
\noindent Creation date: 06/26/2021\\
\noindent Revised: Jonathan Blake (6/27/2021)\\
