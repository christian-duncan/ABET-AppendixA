\label{MA150}  % Replace course code with ID - e.g. \label{CSC110}
\begin{enumerate}[1.]
\item Course number and name\\
  {\bfseries
    % Add course number and name
    % --------------------------
    MA150, Integral Calculus With Applications

  }
  
\item Credits and contact hours\\
  {\bfseries
    % Add Number of Credits and Contact Hours
    % ---------------------------------------
    1  % Default, change if needed
  }

\item Instructor's or course coordinator's name\\
  {\bfseries
    % Add Instructor name
    % -------------------
    Cornelius Nelan
  }

\item Text book, title, author, and year\\
  {\bfseries
    % Add Textbook info
    % --------------------------
    {\em Calculus, 8th edition}, James Stewart, 2015
  }
\begin{enumerate}[a.]
\item Other supplemental materials\\
  {\bfseries
    % List any other supplemental material
    % ------------------------------------
    None    
  }
\end{enumerate}

\item Specific course information
\begin{enumerate}[a.]  
\item Catalog description of the content of the course\\
  {\bfseries
    % Add catalog description
    % -----------------------
his course provides a bridge from MA 141 to MA 152 or MA 153. Students review basic integration rules, integration by substitution, The Fundamental Theorem of Calculus, numerical integration and applications of integration, including area between curves, volumes, arc length and applications from physics. A graphing calculator is required; the TI-83 or TI-84 is recommended.    
  }

\item prerequisites or co-requisites\\
  {\bfseries
    % List any pre/co requisites
    % --------------------------
    % None   % Uncomment if none
    Prerequisites: MA141 (Minimum Grade C-) \\  % Just some defaults
    % Corequisites: 
  }

\item indicate whether a required, elective, or selected elective\\ % (as per Table 5-1)
  {\bfseries
    % R, E, or S (uncomment)
    % --------------------------
    % Required
    % Elective
    Selected elective
  }

\end{enumerate}

\item Specific goals for the course
\begin{enumerate}
\item specific outcomes of instruction\\ % , for example, ``The student will be able to explain the significance of current research about a particular topic.'' \\
  {\bfseries
    % 
    % --------------------------
    Course Learning Objectives (the student will be able to):
    \begin{enumerate}[1.]
    \item Develop an appreciation of the power of calculus as a tool for modeling reality.
    \item Develop geometric intuition and deductive skill.
    \item Appreciate the concept of integration and its various applications
    \end{enumerate}
  }

\item explicitly indicate which of the student outcomes listed in Criterion 3 or any other outcomes are addressed by the course.\\
  {\bfseries
    None
    % C1 (CLO ...),
    % C2 (CLO ...),
    % C3 (CLO ...),
    % C4 (CLO ...),
    % C5 (CLO ...),
    % C6 (CLO ...)
  }
\end{enumerate}

\item Brief list of topics to be covered\\
  {\bfseries
    Lecture Topics
    \begin{itemize}
      \item Integrals and Antiderivatives: u substitution
      \item Areas and integrals
      \item Fundamental Theorem of Calculus
      \item Approximating Sums
      \item Integral as a limit
      \item Interpretations and applications
      \item Volumes
      \item Work
      \item Average Value
      \item Arc Length
      \item Surface Area
      \item Engineering Applications
    \end{itemize}
  }

\end{enumerate}

\noindent Prepared by: Christian Duncan\\
\noindent Creation date: 06/29/2021\\
\noindent Revised:\\
