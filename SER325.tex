\label{SER325}  % Replace course code with ID - e.g. \label{CSC110}
\begin{enumerate}[1.]
\item Course number and name\\
  {\bfseries
    % Add course number and name
    % --------------------------
    CSC/SER325, Database Systems
  }

\item Credits and contact hours\\
  {\bfseries
    % Add Number of Credits and Contact Hours
    % ---------------------------------------
    3
  }

\item Instructor's or course coordinator's name\\
  {\bfseries
    % Add Instructor name
    % -------------------
    Ruby ElKharboutly
  }

\item Text book, title, author, and year\\
  {\bfseries
    % Add Textbook info
    % --------------------------
	No Textbook

  }
\begin{enumerate}[a.]
\item Other supplemental materials\\
  {\bfseries
    % List any other supplemental material
    % ------------------------------------
    None
  }
\end{enumerate}

\item Specific course information
\begin{enumerate}[a.]
\item Catalog description of the content of the course\\
  {\bfseries
    % Add catalog description
    % -----------------------
    Students are introduced to the theory and application of database systems. Topics include data modeling and the relational model, query languages, relational database design, transaction processing, databases and physical database design.
    }
\item prerequisites or co-requisites\\
  {\bfseries
    % List any pre/co requisites
    % --------------------------
    % None   % Uncomment if none
    Prerequisites: CSC/SER 225 and CSC 215 (Note: Both require a minimum grade of C-)
    % Corequisites:
  }

\item indicate whether a required, elective, or selected elective\\ % (as per Table 5-1)
  {\bfseries
    % R, E, or S (uncomment)
    % --------------------------
    Required
    % Elective
    % Selected elective
  }

\end{enumerate}

\item Specific goals for the course
\begin{enumerate}
\item specific outcomes of instruction\\ % , for example, ``The student will be able to explain the significance of current research about a particular topic.'' \\
  {\bfseries
    %
    % --------------------------
    Course Learning Objectives (the student will be able to):
\begin{enumerate}
\item Describe the basic principles of the relational data model.
\item Create a relational schema from a conceptual model developed using the entity-relationship process
\item Apply Normalization techniques to refine a given database schema
\item Use relational algebra to write relational queries
\item Extract information from a database using a declarative query language
\item Develop a data-driven app that performs CRUD operations
\item Implement a relational schema and develop an application to test it
\item Discuss the significance, the theory and algorithm used in creating database indexes
\end{enumerate}
  }

\item explicitly indicate which of the student outcomes listed in Criterion 3 or any other outcomes are addressed by the course.\\
  {\bfseries
    C1 (CLO 1, 2, 3. 5, 4 ,8),
    C2 (CLO 6, 7),
    C3 (CLO 1, 6, 7),
  }
\end{enumerate}

\item Brief list of topics to be covered\\
  {\bfseries
    Lecture Topics
    \begin{itemize}
	\item Into to SQL, DB Dev life cycle, Requirements
	\item DB Design
	\item Relational Data Model
	\item SQL, DB Normalization
	\item Relational Algebra
	\item Data visutalization
	\item index architecture
	\item 	Data models

    \end{itemize}
  }

\end{enumerate}

\noindent Prepared by: Ruby ElKharboutly\\
\noindent Creation date: 06/28/2021\\
\noindent Revised:\\
