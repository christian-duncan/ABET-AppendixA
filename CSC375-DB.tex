\label{CSC375-DB}  % Replace course code with ID - e.g. \label{CSC110}
\begin{enumerate}[1.]
\item Course number and name\\
  {\bfseries
    % Add course number and name
    % --------------------------
    CSC375, Advanced Database Programming
  }

\item Credits and contact hours\\
  {\bfseries
    % Add Number of Credits and Contact Hours
    % ---------------------------------------
    3
  }

\item Instructor's or course coordinator's name\\
  {\bfseries
    % Add Instructor name
    % -------------------
    Prof. Dennis Klemenz
  }

\item Text book, title, author, and year\\
  {\bfseries
    % Add Textbook info
    % --------------------------
    No required textbook
  }
\begin{enumerate}[a.]
\item Other supplemental materials\\
  {\bfseries
    % List any other supplemental material
    % ------------------------------------
    Recommended books:

    SQL The Complete Reference, Weinerg, Third Edition. McGraw-Hill\\
    SQL Cookbook, Molinaro, First Edition. O’Reilly\\
    SQL For Smarties, Celko, Fifth Edition. Morgan Kaufman. (Book series)
  }
\end{enumerate}

\item Specific course information
\begin{enumerate}[a.]
\item Catalog description of the content of the course\\
  {\bfseries
    % Add catalog description
    % -----------------------
General Catalog Description: This course explores advanced computer science topics not available in other courses, as well as new topics as they emerge in this rapidly evolving discipline. Topics may be interdisciplinary.

Specific Course Description: Not available
  }

\item prerequisites or co-requisites\\
  {\bfseries
    % List any pre/co requisites
    % --------------------------
    % None   % Uncomment if none
    Prerequisites: CSC215 and SER225 (Minimum Grade C-)
    % Corequisites:
  }

\item indicate whether a required, elective, or selected elective\\ % (as per Table 5-1)
  {\bfseries
    % R, E, or S (uncomment)
    % --------------------------
    % Required
    % Elective
    Selected elective
  }

\end{enumerate}

\item Specific goals for the course
\begin{enumerate}
\item specific outcomes of instruction\\ % , for example, ``The student will be able to explain the significance of current research about a particular topic.'' \\
  {\bfseries
    %
    % --------------------------
    Course Learning Outcomes (the student will know):
\begin{enumerate}
\item The basic terminology of relational databases
\item How to write meaningful, accurate, formatted and optimized SQL code
\item How to CRUD (create, retrieve, update and delete) data
\item How a DBMS interprets SQL code
\item How to optimize SQL code
\item How to create a SQL script, procedures, triggers and custom aggregation functions
\item How to visualize data
\end{enumerate}
  }

\item explicitly indicate which of the student outcomes listed in Criterion 3 or any other outcomes are addressed by the course.\\
  {\bfseries
    None
    % C1,
    % C2,
    % C3,
    % C4,
    % C5,
    % C6
  }
\end{enumerate}

\item Brief list of topics to be covered\\
  {\bfseries
    Not provided
  }

\end{enumerate}

\noindent Prepared by: Christian Duncan\\
\noindent Creation date: 06/28/2021\\
\noindent Revised: Jonathan Blake (06/28/2021)\\
