\label{CSC375-ML}  % Replace course code with ID - e.g. \label{CSC110}
\begin{enumerate}[1.]
\item Course number and name\\
  {\bfseries
    % Add course number and name
    % --------------------------
    CSC375, Machine Learning
  }

\item Credits and contact hours\\
  {\bfseries
    % Add Number of Credits and Contact Hours
    % ---------------------------------------
    3
  }

\item Instructor's or course coordinator's name\\
  {\bfseries
    % Add Instructor name
    % -------------------
    Prof. Emre Tokgoz
  }

\item Text book, title, author, and year\\
  {\bfseries
    % Add Textbook info
    % --------------------------
    Deep Learning with PyTorch by Eli Stevens, Luca Antiga, and Thomas Viehmann, Manning Publications, 2020. ISBN 9781617295263
  }
\begin{enumerate}[a.]
\item Other supplemental materials\\
  {\bfseries
    % List any other supplemental material
    % ------------------------------------
    Supplemental material provided on Blackboard
  }
\end{enumerate}

\item Specific course information
\begin{enumerate}[a.]
\item Catalog description of the content of the course\\
  {\bfseries
    % Add catalog description
    % -----------------------
General Catalog Description: This course explores advanced computer science topics not available in other courses, as well as new topics as they emerge in this rapidly evolving discipline. Topics may be interdisciplinary.

Specific Course Description: Students are initially introduced to the fundamentals of Python programming. Machine Learning concepts
are covered by using Jupyter Notebook and Python with emphasis given to the use of PyTorch library with applications.
Students are expected to work in groups to explain theoretical Machine Learning concepts with real-life applications and
strategize and solve problems. Participants are also introduced to basic bioinformatics concepts by using Biopython. Healthcare
applications are emphasized throughout the course. The completion of a real-life semester project in groups is an essential
component of grading and participation.
  }

\item prerequisites or co-requisites\\
  {\bfseries
    % List any pre/co requisites
    % --------------------------
    % None   % Uncomment if none
    Prerequisites: CSC215 and SER225 (Minimum Grade C-)
    % Corequisites:
  }

\item indicate whether a required, elective, or selected elective\\ % (as per Table 5-1)
  {\bfseries
    % R, E, or S (uncomment)
    % --------------------------
    % Required
    % Elective
    Selected elective
  }

\end{enumerate}

\item Specific goals for the course
\begin{enumerate}
\item specific outcomes of instruction\\ % , for example, ``The student will be able to explain the significance of current research about a particular topic.'' \\
  {\bfseries
    %
    % --------------------------
    Course Learning Outcomes (the student will be able to):
\begin{enumerate}
\item Apply Python programming language in healthcare.
\item Function effectively in teams to explain theoretical machine learning concepts.
\item Use Machine Learning for meaningful data analysis.
\item Fulfill ethical and professional responsibilities by working in groups on projects.
\item Use PyTorch for Machine Learning image analysis.
\item Design software solutions by using Python.
\item Complete a real-life project by working in groups and applying machine learning concepts.
\end{enumerate}
  }

\item explicitly indicate which of the student outcomes listed in Criterion 3 or any other outcomes are addressed by the course.\\
  {\bfseries
    % None
    C1,
    C2,
    C3,
    C4,
    C5,
    C6
  }
\end{enumerate}

\item Brief list of topics to be covered\\
  {\bfseries
    Lecture Topics
    \begin{itemize}
    \item Software Fundamentals
    \item Machine Learning Fundamentals using Jupyter Notebook
    \item Coverage of Machine Learning techniques and fundamental theoretical knowledge needed to cover Deep Learning with
PyTorch
    \item Team Presentations of Book Chapters
    \item Introduction to Biopython and Fundamentals of Bioinformatics
    \end{itemize}
  }

\end{enumerate}

\noindent Prepared by: Christian Duncan\\
\noindent Creation date: 06/28/2021\\
\noindent Revised: Jonathan Blake (06/28/2021)\\
