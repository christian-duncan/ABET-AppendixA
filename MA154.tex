\label{MA154}  % Replace course code with ID - e.g. \label{CSC110}
\begin{enumerate}[1.]
\item Course number and name\\
  {\bfseries
    % Add course number and name
    % --------------------------
    MA154, Calculus II: Part B
  }
  
\item Credits and contact hours\\
  {\bfseries
    % Add Number of Credits and Contact Hours
    % ---------------------------------------
    2  % Default, change if needed
  }

\item Instructor's or course coordinator's name\\
  {\bfseries
    % Add Instructor name
    % -------------------
    Cornelius Nelan    
  }

\item Text book, title, author, and year\\
  {\bfseries
    % Add Textbook info
    % --------------------------
    {\em Early Transcendental Functions: Edition 7e}, Larson and Edwards.     
  }
\begin{enumerate}[a.]
\item Other supplemental materials\\
  {\bfseries
    % List any other supplemental material
    % ------------------------------------
    Access to WebAssign and a graphing calculator (TI-83 or TI-84 recommended)
  }
\end{enumerate}

\item Specific course information
\begin{enumerate}[a.]  
\item Catalog description of the content of the course\\
  {\bfseries
    % Add catalog description
    % -----------------------
In this course students study differential equations, conic sections, parametric equations, polar coordinates, vectors, operations on vectors, lines and planes in space, three-dimensional coordinate systems (cylindrical and spherical coordinates) and quadric surfaces. Offered the second half of each semester.
}

\item prerequisites or co-requisites\\
  {\bfseries
    % List any pre/co requisites
    % --------------------------
    % None   % Uncomment if none
    Prerequisites: MA151 or MA150 (Minimum Grade C-)\\ % Just some defaults
    Corequisites: Take MA153.
  }

\item indicate whether a required, elective, or selected elective\\ % (as per Table 5-1)
  {\bfseries
    % R, E, or S (uncomment)
    % --------------------------
    % Required
    % Elective
    Selected elective
  }

\end{enumerate}

\item Specific goals for the course
\begin{enumerate}
\item specific outcomes of instruction\\ % , for example, ``The student will be able to explain the significance of current research about a particular topic.'' \\
  {\bfseries
    % 
    % --------------------------
    Course Learning Objectives (the student will be able to):
    \begin{enumerate}[1.]
    \item Analyze equations and properties of parabolas, ellipses, and hyperbolas.
    \item To understand the polar coordinate system.  
    \item Rewrite rectangle coordinates into parametric and polar coordinates and vice versa.
    \item To find different ways to describe curves in two and three dimensions: parametric equations, polar coordinates.
    \item To use parametric and polar coordinates to compute area and arc length.
    \item To solve first order differential equations in one variable by observation, separation of variables, and other techniques.
    \item To use differential equations to model real-life problems and solve them: growth, decay, etc.
    \item To understand vectors in two and three dimensions.
    \item To use various operations to analyze vectors: dot product, scalar product, cross product, norms, projections, etc.
    \item To write the equation for lines and planes in space.
    \end{enumerate}
  }

\item explicitly indicate which of the student outcomes listed in Criterion 3 or any other outcomes are addressed by the course.\\
  {\bfseries
    None
    % C1 (CLO ...),
    % C2 (CLO ...),
    % C3 (CLO ...),
    % C4 (CLO ...),
    % C5 (CLO ...),
    % C6 (CLO ...)
  }
\end{enumerate}

\item Brief list of topics to be covered\\
  {\bfseries
    Lecture Topics
    \begin{itemize}
    \item Review of MA 153;
    \item Conic sections: parabolas, ellipses, and hyperbolas;
    \item Parametric Equations
    \item Polar Coordinates, Applications of Polar Coordinates to area and arc length
    \item General and Specific solutions to differential equations, Slope fields and Euler’s method, Applications: Separation of variables, Linear Differential equations
    \item Vectors in two and three dimensions: Lines and planes in space.
    \item Operations on vectors: dot product, cross product.
    \item Cylindrical and Spherical Coordinates (time permitting) 
    \end{itemize}
  }
\end{enumerate}

\noindent Prepared by: Christian Duncan\\
\noindent Creation date: 06/29/2021\\
\noindent Revised:\\
