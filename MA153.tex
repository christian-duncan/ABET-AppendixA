\label{MA153}  % Replace course code with ID - e.g. \label{CSC110}
\begin{enumerate}[1.]
\item Course number and name\\
  {\bfseries
    % Add course number and name
    % --------------------------
    MA153, Calculus II: Part A
  }
  
\item Credits and contact hours\\
  {\bfseries
    % Add Number of Credits and Contact Hours
    % ---------------------------------------
    2  % Default, change if needed
  }

\item Instructor's or course coordinator's name\\
  {\bfseries
    % Add Instructor name
    % -------------------
    Cornelius Nelan    
  }

\item Text book, title, author, and year\\
  {\bfseries
    % Add Textbook info
    % --------------------------
    {\em Early Transcendental Functions: Edition 7e}, Larson and Edwards, 2019
  }
\begin{enumerate}[a.]
\item Other supplemental materials\\
  {\bfseries
    % List any other supplemental material
    % ------------------------------------
    None    
  }
\end{enumerate}

\item Specific course information
\begin{enumerate}[a.]  
\item Catalog description of the content of the course\\
  {\bfseries
    % Add catalog description
    % -----------------------
Students in this course study techniques of integration and infinite sequences and series. Techniques studied include u-substitution, integrals involving logarithms and inverse trigonometric functions, trigonometric integrals, trigonometric substitution, integration by parts, and partial fractions. For infinite series, the course includes a study of convergence, tests of convergence, power series, and Taylor and Maclaurin series. Additional topics include indeterminate forms, L'Hopital's Rule, and improper integrals. Offered the first half of each semester.    
  }

\item prerequisites or co-requisites\\
  {\bfseries
    % List any pre/co requisites
    % --------------------------
    % None   % Uncomment if none
    Prerequisites: MA141 or MA151 (Minimum Grade C-) % Just some defaults
    % Corequisites: 
  }

\item indicate whether a required, elective, or selected elective\\ % (as per Table 5-1)
  {\bfseries
    % R, E, or S (uncomment)
    % --------------------------
    % Required
    % Elective
    Selected elective
  }

\end{enumerate}

\item Specific goals for the course
\begin{enumerate}
\item specific outcomes of instruction\\ % , for example, ``The student will be able to explain the significance of current research about a particular topic.'' \\
  {\bfseries
    % 
    % --------------------------
    Course Learning Objectives
    \begin{enumerate}[1.]
    \item To continue exploration of the concept of derivative and integral in single variable calculus
    \item To problem solve using various techniques to compute anti-derivatives
    \item To approximate integrals using numerical methods.
    \item To study calculus of the infinite using improper integrals and infinite sums.
    \item To study limits using L’Hospital’s Rule
    \item To study sequences and series.
    \item To learn numerical methods of computing values of transcendental functions like sin x, using Taylor series.
    \item To explain how calculators and computers compute esoteric values.
    \item To lay the foundation for further study in calculus, physics, and engineering.
    \end{enumerate}
  }

\item explicitly indicate which of the student outcomes listed in Criterion 3 or any other outcomes are addressed by the course.\\
  {\bfseries
    None
    % C1 (CLO ...),
    % C2 (CLO ...),
    % C3 (CLO ...),
    % C4 (CLO ...),
    % C5 (CLO ...),
    % C6 (CLO ...)
  }
\end{enumerate}

\item Brief list of topics to be covered\\
  {\bfseries
    Lecture Topics
    \begin{itemize}
    \item review of derivatives,
    \item integrals, antiderivatives, u-substitutions.
    \item integration by parts, trigonometric integrals
    \item trigonometric substitutions,
    \item integration using partial fractions,
    \item numerical methods
    \item improper integrals,
    \item L'Hopital’s Rule
    \item sequences and series, convergence and divergence, tests for convergence
    \item power series, Taylor series, numerical calculations
    \end{itemize}
  }

\end{enumerate}

\noindent Prepared by: Christian Duncan\\
\noindent Creation date: 06/29/2021\\
\noindent Revised:\\
