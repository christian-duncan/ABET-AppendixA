\label{CSC107}  % Replace course code with ID - e.g. \label{CSC110}
\begin{enumerate}[1.]
\item Course number and name\\
  {\bfseries
    % Add course number and name
    % --------------------------
    CSC107, Structured Programming Techniques
  }
  
\item Credits and contact hours\\
  {\bfseries
    % Add Number of Credits and Contact Hours
    % ---------------------------------------
    1  % Default, change if needed
  }

\item Instructor's or course coordinator's name\\
  {\bfseries
    % Add Instructor name
    % -------------------
    Christian Duncan (Coordinator)    
  }

\item Text book, title, author, and year\\
  {\bfseries
    % Add Textbook info
    % --------------------------
    None.
  }
\begin{enumerate}[a.]
\item Other supplemental materials\\
  {\bfseries
    % List any other supplemental material
    % ------------------------------------
    None. Blackboard (LMS) pages link to various videos, short tutorials, readings, etc. 
  }
\end{enumerate}

\item Specific course information
\begin{enumerate}[a.]  
\item Catalog description of the content of the course\\
  {\bfseries
    % Add catalog description
    % -----------------------
The main purpose of this course is to fill any gaps between the Programming and Problem Solving course (CSC 110) and
the Introduction to Programming for Engineers course (CSC 106) or similar introduction to programming course.
Topics include a basic programming refresher, binary number representation, debugging strategies and simple recursion.
  }

\item prerequisites or co-requisites\\
  {\bfseries
    % List any pre/co requisites
    % --------------------------
    % None   % Uncomment if none
    Prerequisites: CSC106 (Minimum Grade C-) % Just some defaults
    % Corequisites: 
  }

\item indicate whether a required, elective, or selected elective\\ % (as per Table 5-1)
  {\bfseries
    % R, E, or S (uncomment)
    % --------------------------
    Required\footnote{Some CSC majors can take this option if starting in Engineering and switching to CSC. CSC106 and CSC107 can be substituted for CSC110 and CSC110L.}
    % Elective
    % Selected elective
  }

\end{enumerate}

\item Specific goals for the course
\begin{enumerate}
\item specific outcomes of instruction\\ % , for example, ``The student will be able to explain the significance of current research about a particular topic.'' \\
  {\bfseries
    % 
    % --------------------------
    Course Learning Objectives (the student will be able to):
    \begin{itemize}
    \item Discuss the challenges inherent in communicating with (programming) a 
(non-reasoning) computer;
%\item Discuss the history of computing, and the role programming paradigms play in problem solving;
       \item Break a complex problem down into smaller more manageable components;
       \item Provide step-by-step instructions to solve small computational problems;
       \item Using at least one programming language\footnote{In our case, the language shall be Java.}, 
       write examples of and solve problems in programs with the following basic programming constructs and tools:
       \begin{enumerate}
       \item mathematical expressions,
       \item conditional expressions,
       \item simple iterative statements,
       \item simple functions,
       \item simple data structures, such as arrays, lists, and strings; and
       \end{enumerate}
       \item Discuss the importance of commenting and good code structure and style.
    \end{itemize}
  }

\item explicitly indicate which of the student outcomes listed in Criterion 3 or any other outcomes are addressed by the course.\\
  {\bfseries
    % None
    C1,
    C2,
    % C3 (CLO ...),
    % C4 (CLO ...),
    % C5 (CLO ...),
    C6
  }
\end{enumerate}

\item Brief list of topics to be covered\\
  {\bfseries
    Lecture Topics (course is broken down into modules)
    \begin{itemize}
    \item Module 0: Reviewing syllabus, policies, and other course materials
    \item Module 1: Setting up the basic Java environment (and a little about Object-Oriented programming)
    \item Module 2: Exploring the nature of storing and computing numbers in binary
    \item Module 3: Beginning to program in Java, the basics of imperative programming
    \item Module 4: Beginning to program in Java, the basics of functional programming
    \item Module 5: Debugging programs
    \item Module 6: Exploring the basic nature of recursion, in mathematics and programming
    \end{itemize}
  }

\end{enumerate}

\noindent Prepared by: Christian Duncan\\
\noindent Creation date: 06/28/2021\\
\noindent Revised:\\
