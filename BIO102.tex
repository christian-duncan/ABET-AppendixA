\label{BIO102}  % Replace course code with ID - e.g. \label{CSC110}
\begin{enumerate}[1.]
\item Course number and name\\
  {\bfseries
    % Add course number and name
    % --------------------------
    BIO 102, General Biology II
  }

\item Credits and contact hours\\
  {\bfseries
    % Add Number of Credits and Contact Hours
    % ---------------------------------------
    3
  }

\item Instructor's or course coordinator's name\\
  {\bfseries
    % Add Instructor name
    % -------------------
    Michael Vieth
  }

\item Text book, title, author, and year\\
  {\bfseries
    % Add Textbook info
    % --------------------------
    Mastering Biology with Pearson, which includes: Audesirk T, Audesirk G, and Byers B. 2016. Biology: Life on Earth with Physiology, 11th edition.  New York: Pearson
  }
\begin{enumerate}[a.]
\item Other supplemental materials\\
  {\bfseries
    % List any other supplemental material
    % ------------------------------------
   None
  }
\end{enumerate}

\item Specific course information
\begin{enumerate}[a.]
\item Catalog description of the content of the course\\
  {\bfseries
    % Add catalog description
    % -----------------------
This course covers the basic concepts of life science with an emphasis on animal anatomy and physiology, animal reproduction and development, the nervous system, evolutionary mechanisms and ecological principles. Selected topics include microevolution, speciation, macroevolution, animal behavior and application of comparative anatomy and physiology to illuminate evolutionary relationships and their ecological context. This course is primarily for students in health science programs or in the School of Engineering. Second semester of a full-year course; must be taken in sequence.
  }

\item prerequisites or co-requisites\\
  {\bfseries
    % List any pre/co requisites
    % --------------------------
    % None   % Uncomment if none
    Prerequisites: BIO101, BIO101L (Minimum Grade C-)\\
    Corequisites: BIO102L
  }

\item indicate whether a required, elective, or selected elective\\ % (as per Table 5-1)
  {\bfseries
    % R, E, or S (uncomment)
    % --------------------------
    % Required
    % Elective
    Selected elective
  }

\end{enumerate}

\item Specific goals for the course
\begin{enumerate}
\item specific outcomes of instruction\\ % , for example, ``The student will be able to explain the significance of current research about a particular topic.'' \\
  {\bfseries
    %
    % --------------------------
    Course Objectives:

    The students will be introduced to the philosophy of science and the basic concepts of life sciences, in particular.  The student will be encouraged to develop library and communication skills, familiarize him/herself with experimental design and the interpretation of biological data and build a foundation for advanced study in biology.  In addition, the student will be encouraged to thoughtfully consider the ethical implications of scientific research.
  }

\item explicitly indicate which of the student outcomes listed in Criterion 3 or any other outcomes are addressed by the course.\\
  {\bfseries
    None
    % C1,
    % C2,
    % C3,
    % C4,
    % C5,
    % C6
  }
\end{enumerate}

\item Brief list of topics to be covered\\
  {\bfseries
    None listed in syllabus
  }

\end{enumerate}

\noindent Prepared by: Christian Duncan\\
\noindent Creation date: 06/26/2021\\
\noindent Revised: Jonathan Blake (06/27/21)\\
