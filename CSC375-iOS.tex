\label{CSC375-iOS}  % Replace course code with ID - e.g. \label{CSC110}
\begin{enumerate}[1.]
\item Course number and name\\
  {\bfseries
    % Add course number and name
    % --------------------------
    CSC375, iOS Programming
  }

\item Credits and contact hours\\
  {\bfseries
    % Add Number of Credits and Contact Hours
    % ---------------------------------------
    3
  }

\item Instructor's or course coordinator's name\\
  {\bfseries
    % Add Instructor name
    % -------------------
    Prof. Dennis Klemenz
  }

\item Text book, title, author, and year\\
  {\bfseries
    % Add Textbook info
    % --------------------------
    None required
  }
\begin{enumerate}[a.]
\item Other supplemental materials\\
  {\bfseries
    % List any other supplemental material
    % ------------------------------------
    None
  }
\end{enumerate}

\item Specific course information
\begin{enumerate}[a.]
\item Catalog description of the content of the course\\
  {\bfseries
    % Add catalog description
    % -----------------------
General Catalog Description: This course explores advanced computer science topics not available in other courses, as well as new topics as they emerge in this rapidly evolving discipline. Topics may be interdisciplinary.

Specific Course Description: iOS Programming is a course that covers the
design and development of mobile apps for Apple devices that run iOS.  The
course will cover the two languages used in iOS Programming (Objective-C and
Swift) and will also introduce design approaches that compare native vs.
compiled/interpreted solutions.  The course will cover topics such as Mobile
Interface \& MVC Design, Cocoa Touch, Multimedia, Camera and a multitude of
other frameworks that are used to design robust mobile apps.  Students will be
required to design a mobile app in the course.
  }

\item prerequisites or co-requisites\\
  {\bfseries
    % List any pre/co requisites
    % --------------------------
    % None   % Uncomment if none
    Prerequisites: CSC215 and SER225 (Minimum Grade C-)
    % Corequisites:
  }

\item indicate whether a required, elective, or selected elective\\ % (as per Table 5-1)
  {\bfseries
    % R, E, or S (uncomment)
    % --------------------------
    % Required
    % Elective
    Selected elective
  }

\end{enumerate}

\item Specific goals for the course
\begin{enumerate}
\item specific outcomes of instruction\\ % , for example, ``The student will be able to explain the significance of current research about a particular topic.'' \\
  {\bfseries
    %
    % --------------------------
    Students will be introduced to:
    \begin{itemize}
      \item iOS Operating System Concepts including Cocoa, Blocks, and Multithreading
      \item Cocoa framework for video, images, sound and touch
      \item Objective-C \& Swift programming languages
      \item Apple’s IDE (Xcode)
      \item Modal, View, Controller (MVC) design
      \item iDevices vs. Other Mobile Platforms (Android \& Blackberry)
      \item Classes and handling events
      \item Deploying apps to a test device or to the App Store
      \item Proper GUI / UX design
    \end{itemize}
  }

\item explicitly indicate which of the student outcomes listed in Criterion 3 or any other outcomes are addressed by the course.\\
  {\bfseries
    None
    % C1,
    % C2,
    % C3,
    % C4,
    % C5,
    % C6
  }
\end{enumerate}

\item Brief list of topics to be covered\\
  {\bfseries
    \begin{itemize}
      \item Intro to iOS \& Xcode
      \item Building a Sample App
      \item Intro to Objective-C \& Swift
      \item Model-View-Controller Concept
      \item Deep Dive into Objective-C
      \item Deep Dive into Swift
      \item Views \& View Controllers
      \item Sizing Views for different devices
      \item Gestures \& Touches
      \item GUI Design Principles
      \item Scenes \& Popovers
      \item Orientation \& Motion
      \item Reading / Writing Persistent Data
      \item Core Media
      \item Security
      \item Background Awareness
      \item Memory Management \& Media
    \end{itemize}
  }

\end{enumerate}

\noindent Prepared by: Jonathan Blake\\
\noindent Creation date: 06/29/2021\\
\noindent Revised:\\
