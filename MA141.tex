\label{MA141}  % Replace course code with ID - e.g. \label{CSC110}
\begin{enumerate}[1.]
\item Course number and name\\
  {\bfseries
    % Add course number and name
    % --------------------------
    MA141, Calculus of a Single Variable I    
  }
  
\item Credits and contact hours\\
  {\bfseries
    % Add Number of Credits and Contact Hours
    % ---------------------------------------
    3  % Default, change if needed
  }

\item Instructor's or course coordinator's name\\
  {\bfseries
    % Add Instructor name
    % -------------------
    Varied    
  }

\item Text book, title, author, and year\\
  {\bfseries
    % Add Textbook info
    % --------------------------
    {\em Single Variable Calculus, Early Transcendentals} (with WebAssign), James Stewart
  }
\begin{enumerate}[a.]
\item Other supplemental materials\\
  {\bfseries
    % List any other supplemental material
    % ------------------------------------
    Calculator: TI-83 or TI-84 (TI-89 and Nspires are not allowed)
  }
\end{enumerate}

\item Specific course information
\begin{enumerate}[a.]  
\item Catalog description of the content of the course\\
  {\bfseries
    % Add catalog description
    % -----------------------
This course covers functions, graphs, limits, continuity, derivatives, applications of derivatives, antiderivatives and definite integrals, as well as the Fundamental Theorem of Calculus. This course significantly advances the following Essential Learning Outcomes: quantitative reasoning, critical thinking and reasoning. Many sections require a TI-83/84 calculator (or the equivalent); check with the instructor. Students cannot receive credit for both MA 141 and MA 151.    
  }

\item prerequisites or co-requisites\\
  {\bfseries
    % List any pre/co requisites
    % --------------------------
    % None   % Uncomment if none
    Prerequisites: MA140 (Minimum Grade C-) or score of 5 on Math Placement  % Just some defaults
    % Corequisites: 
  }

\item indicate whether a required, elective, or selected elective\\ % (as per Table 5-1)
  {\bfseries
    % R, E, or S (uncomment)
    % --------------------------
    Required
    % Elective
    % Selected elective
  }

\end{enumerate}

\item Specific goals for the course
\begin{enumerate}
\item specific outcomes of instruction\\ % , for example, ``The student will be able to explain the significance of current research about a particular topic.'' \\
  {\bfseries
    % 
    % --------------------------
    Not specified.
    %% Course Learning Objectives (the student will be able to):
    %% \begin{enumerate}[1.]
    %%   \item TBD
    %% \end{enumerate}
  }

\item explicitly indicate which of the student outcomes listed in Criterion 3 or any other outcomes are addressed by the course.\\
  {\bfseries
    None
    % C1 (CLO ...),
    % C2 (CLO ...),
    % C3 (CLO ...),
    % C4 (CLO ...),
    % C5 (CLO ...),
    % C6 (CLO ...)
  }
\end{enumerate}

\item Brief list of topics to be covered\\
  {\bfseries
    Lecture Topics
    \begin{itemize}
      \item Functions 
      \item Essential functions and operations
      \item Exponential and Logarithmic Functions
      \item The limit of a function
      \item Limit laws and continuity
      \item Limits at infinity
      \item Derivatives and Rates of Change
      \item The Derivative as a function
      \item Derivative Rules
      \item Product and Quotient Rule
      \item Derivative of the Trigonometric Functions
      \item The Chain Rule
      \item Implicit Differentiation
      \item Derivative of the Logarithmic Functions
      \item Max and min values; How derivatives shape Graphs
      \item Optimization
      \item Related Rates
      \item Anti-derivatives
      \item Sigma Notation
      \item Areas and Distances
      \item The Definite Integral
      \item The Fundamental Theorem of Calculus
      \item Indefinite Integrals
      \item The Substitution Rule
\end{itemize}
  }

\end{enumerate}

\noindent Prepared by: Christian Duncan\\
\noindent Creation date: 06/29/2021\\
\noindent Revised:\\
