\label{CHE110}  % Replace course code with ID - e.g. \label{CSC110}
\begin{enumerate}[1.]
\item Course number and name\\
  {\bfseries
    % Add course number and name
    % --------------------------
CHE 110, General Chemistry I
  }

\item Credits and contact hours\\
  {\bfseries
    % Add Number of Credits and Contact Hours
    % ---------------------------------------
3
  }

\item Instructor's or course coordinator's name\\
  {\bfseries
    % Add Instructor name
    % -------------------
    Dr. James F. Kirby, Ph. D.
  }

\item Text book, title, author, and year\\
  {\bfseries
    % Add Textbook info
    % --------------------------
    General Chemistry by D. Ebbing \& S. Gammon, 11th ed., Cengage Learning, 2017.
  }
\begin{enumerate}[a.]
\item Other supplemental materials\\
  {\bfseries
    % List any other supplemental material
    % ------------------------------------
    Cengage Unlimited, OWLv2, Calculator supporting logarithmic functions
  }
\end{enumerate}

\item Specific course information
\begin{enumerate}[a.]
\item Catalog description of the content of the course\\
  {\bfseries
    % Add catalog description
    % -----------------------
Students study the atomic theory of matter, nomenclature, chemical formulas and reaction equations, stoichiometry, the gas laws and the kinetic molecular theory, thermochemistry, atomic structure, periodicity of the elements, chemical bonding and molecular structure. (Note: this course is designed for science majors.)
  }

\item prerequisites or co-requisites\\
  {\bfseries
    % List any pre/co requisites
    % --------------------------
    % None   % Uncomment if none
    A math placement score of 3 or higher (or suitable math course) is required to enroll in CHE110. Corequisites: CHE110L
  }

\item indicate whether a required, elective, or selected elective\\ % (as per Table 5-1)
  {\bfseries
    % R, E, or S (uncomment)
    % --------------------------
    % Required
    % Elective
    Selected elective
  }

\end{enumerate}

\item Specific goals for the course
\begin{enumerate}
\item specific outcomes of instruction\\ % , for example, ``The student will be able to explain the significance of current research about a particular topic.'' \\
  {\bfseries
    %
    % --------------------------
    Outcomes are given as a list of fundamental principles and concepts (see below).
  }

\item explicitly indicate which of the student outcomes listed in Criterion 3 or any other outcomes are addressed by the course.\\
  {\bfseries
    None
    % C1,
    % C2,
    % C3,
    % C4,
    % C5,
    % C6
  }
\end{enumerate}

\item Brief list of topics to be covered\\
  {\bfseries
    Fundamental Principles and Concepts
    \begin{itemize}
      \item Physical and chemical properties of matter
      \item S.I. units and their application in dimensional analysis
      \item Density
      \item Proper use of significant figures
      \item Structure of the atom, atomic number, atomic mass and isotopes
      \item Significance of the periodic table and its use to predict the formation of molecules and compounds
      \item Chemical formulas of compounds and how they are named
      \item Mass relationships in chemical reactions and the Law of Conservation of Matter
      \item Balance chemical equations and use them to calculate the amount of product formed from a given amount of reactants (stoichiometry)
      \item Solution behavior of electrolytes and non-electrolytes
      \item Chemical reactions in aqueous solutions including acid-base reactions, oxidation and reduction reactions and precipitation reactions
      \item Express the concentration of a solution using molarity and dilute a solution to obtain a desired concentration
      \item Gas behavior based on the Kinetic Molecular Model
      \item Thermochemistry and the Law of Conservation of Energy as applied to chemical systems
      \item Energy changes in chemical reactions
      \item Quantum theory and the electronic structure of the atom
      \item Electron configuration and the Aufbau Principle
      \item Periodic classification of the elements and periodic variation in their physical properties
      \item Basic concepts of chemical bonding in both ionic compounds and covalent molecules
      \item Lewis structures and molecular geometries and polarities of molecules based on application of the VSEPR model
    \end{itemize}
  }

\end{enumerate}

\noindent Prepared by: Christian Duncan\\
\noindent Creation date: 06/26/2021\\
\noindent Revised: Jonathan Blake (6/27/2021)\\
