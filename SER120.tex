\label{SER120}  % Replace course code with ID - e.g. \label{CSC110}
\begin{enumerate}[1.]
\item Course number and name\\
  {\bfseries
    % Add course number and name
    % --------------------------
    SER120, Object-oriented Design and Development
  }

\item Credits and contact hours\\
  {\bfseries
    % Add Number of Credits and Contact Hours
    % ---------------------------------------
    3
  }

\item Instructor's or course coordinator's name\\
  {\bfseries
    % Add Instructor name
    % -------------------
    Ruby ElKharboutly
  }

\item Text book, title, author, and year\\
  {\bfseries
    % Add Textbook info
    % --------------------------


	Object-Oriented Programming in Java: A Graphical Approach by Kathryn E. Sanders \& Andries van Dam
	Addison Wesley, 2006
	ISBN 0-321-24574-1
  }
\begin{enumerate}[a.]
\item Other supplemental materials\\
  {\bfseries
    % List any other supplemental material
    % ------------------------------------
    None
  }
\end{enumerate}

\item Specific course information
\begin{enumerate}[a.]
\item Catalog description of the content of the course\\
  {\bfseries
    % Add catalog description
    % -----------------------
    This course serves as an introduction to the principles of design and programming of software using object-oriented techniques such as inheritance, polymorphism, encapsulation and information hiding. Students will learn event-driven programming and the use of graphics. Additional topics include collections, design patterns and frameworks   }

\item prerequisites or co-requisites\\
  {\bfseries
    % List any pre/co requisites
    % --------------------------
    % None   % Uncomment if none
    Prerequisites: CSC110 , CSC110L (Minimum grade of C-)
    % Corequisites:
  }

\item indicate whether a required, elective, or selected elective\\ % (as per Table 5-1)
  {\bfseries
    % R, E, or S (uncomment)
    % --------------------------
    Required
    % Elective
    % Selected elective
  }

\end{enumerate}

\item Specific goals for the course
\begin{enumerate}
\item specific outcomes of instruction\\ % , for example, ``The student will be able to explain the significance of current research about a particular topic.'' \\
  {\bfseries
    %
    % --------------------------
    Course Learning Objectives (the student will be able to):
\begin{enumerate}
\item Analyze, design and implement software applications using OOP techniques;
	\begin{enumerate}
	\item Use UML class diagrams for designing OO Java applications;
	\item Use OOP concepts such as abstraction, encapsulation, inherence and polymorphism during design and development;
	\item Explain a subset of design patterns and use patterns during design and development;
	\end{enumerate}
\item Design and develop event-driven graphical java applications;
\item Use advanced java APIs and collections;
\end{enumerate}
  }

\item explicitly indicate which of the student outcomes listed in Criterion 3 or any other outcomes are addressed by the course.\\
  {\bfseries
    C1 (CLO 1, 2, 3),
    C2 (CLO 1, 2, 3),
    C3 (CLO 2,3,4),
  }
\end{enumerate}

\item Brief list of topics to be covered\\
  {\bfseries
    Lecture Topics
    \begin{itemize}
      \item Design Methodologies
      \item Objects and Properties
	  \item Methods, parameters and return types
	  \item Inheritance and abstract classes
	  \item Interfaces
	  \item Polymorphism/ code craft
	  \item AWT and Swing
	  \item GUI applications
	  \item Design Patterns, Images, Sound
	  \item  Exceptions, Collections
	  \item Advanced Java
    \end{itemize}
  }

\end{enumerate}

\noindent Prepared by: Ruby ElKharboutly\\
\noindent Creation date: 06/28/2021\\
\noindent Revised:\\
