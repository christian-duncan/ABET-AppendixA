\label{SER120}  % Replace course code with ID - e.g. \label{CSC110}
\begin{enumerate}[1.]
\item Course number and name\\
  {\bfseries
    % Add course number and name
    % --------------------------
SER 120, Object Oriented Programming and Design
  }

\item Credits and contact hours\\
  {\bfseries
    % Add Number of Credits and Contact Hours
    % ---------------------------------------
3
  }

\item Instructor's or course coordinator's name\\
  {\bfseries
    % Add Instructor name
    % -------------------
Prof. Ruby El Kharboutly
  }

\item Text book, title, author, and year\\
  {\bfseries
    % Add Textbook info
    % --------------------------
    Object-Oriented Programming in Java: A Graphical Approach by Kathryn E.
    Sanders \& Andries van Dam, Addison Wesley, 2006
  }
\begin{enumerate}[a.]
\item Other supplemental materials\\
  {\bfseries
    % List any other supplemental material
    % ------------------------------------
    \begin{itemize}
    \item Java How to Program, 9th Edition, by Paul Dietel and Harvey Deitel
    \item Effective Java (free book):
    \begin{verbatim}
      http://dl.softgozar.com/Files/Ebook/Effective_Java_Second_Edition_Softgozar.com.pdf
    \end{verbatim}
  \end{itemize}
  }
\end{enumerate}

\item Specific course information
\begin{enumerate}[a.]
\item Catalog description of the content of the course\\
  {\bfseries
    % Add catalog description
    % -----------------------
This course serves as an introduction to the principles of design and programming of software using
object-oriented techniques such as inheritance, polymorphism, encapsulation and information hiding.
Students will learn event-driven programming and the use of graphics. Additional topics include
collections, design patterns and frameworks
  }

\item prerequisites or co-requisites\\
  {\bfseries
    % List any pre/co requisites
    % --------------------------
    % None   % Uncomment if none
    Prerequisites: CSC 110 and CSC110L with a a minimum grade of C-

    Corequisite: SER 120L
  }

\item indicate whether a required, elective, or selected elective\\ % (as per Table 5-1)
  {\bfseries
    % R, E, or S (uncomment)
    % --------------------------
    % Required
    % Elective
    Required
  }

\end{enumerate}

\item Specific goals for the course
\begin{enumerate}
\item specific outcomes of instruction\\ % , for example, ``The student will be able to explain the significance of current research about a particular topic.'' \\
  {\bfseries
    %
    % --------------------------
    Learning Objectives (the student will be able to):
    \begin{itemize}
      \item Analyze, design and implement software applications using OOP techniques.
      \begin{itemize}
        \item Use UML class diagrams for designing OO Java applications.
        \item Use OOP concepts such as abstraction, encapsulation, inherence and polymorphism during design and development.
        \item Explain a subset of design patterns and use patterns during design and development
      \end{itemize}
      \item Design and develop event-driven graphical java applications.
      \item Use advanced java APIs and collections.
    \end{itemize}
  }

\item explicitly indicate which of the student outcomes listed in Criterion 3 or any other outcomes are addressed by the course.\\
  {\bfseries
    C1,
    C2,
    % C3,
    % C4,
    % C5,
    C6
  }
\end{enumerate}

\item Brief list of topics to be covered\\
  {\bfseries
    Lecture Topics
    \begin{itemize}
      \item Design Methodologies
      \item Objects and Properties
      \item Methods, parameters and return types
      \item Inheritance and abstract classes
      \item Interfaces
      \item Polymorphism/ code craft
      \item AWT and Swing
      \item GUI applications
      \item Design Patterns, Images, Sound
      \item Exceptions, Collections
      \item Advanced Java
    \end{itemize}
  }

\end{enumerate}

\noindent Prepared by: Jonathan Blake\\
\noindent Creation date: 06/29/2021\\
\noindent Revised:\\
